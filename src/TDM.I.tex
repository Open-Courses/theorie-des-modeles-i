%        File: TDM.I.tex
%     Created: Son Mai 31 07:00  2015 C
% Last Change: Son Mai 31 07:00  2015 C
%
\documentclass[a4paper, 12pt]{article}

\usepackage[french]{babel}
\usepackage[T1]{fontenc}
\usepackage[utf8]{inputenc}
\usepackage{hyperref}

\usepackage{amsmath}
\usepackage{amsthm}
\usepackage{amssymb}
\usepackage{physics}

\usepackage{amsmath}

%-------------------------------------------------------------------------------
\newcommand{\naturel}{\mathbb{N}}
\newcommand{\integer}{\mathbb{Z}}
\newcommand{\rational}{\mathbb{Q}}
\renewcommand{\real}{\mathbb{R}} %Already defined in physics package
\newcommand{\complex}{\mathbb{C}}
%-------------------------------------------------------------------------------

%Logical
\def\implies{\Rightarrow}
\def\equiv{\Leftrightarrow}
%-------------------------------------------------------------------------------

%Set theory
\def\union{\cup} %Union
\def\inter{\cap} %Intersection
\newcommand{\comp}[1]{#1^{c}} %Complementary
\def\cartprod{\cross}
\newcommand{\cardinal}[1]{\#(#1)}
%-------------------------------------------------------------------------------

%Topology
\def\interior{\mathring}
\def\adh{\overline}
%-------------------------------------------------------------------------------

%Algebra

% Linear algebra

\newcommand{\matrixSpace}[2]{M_{#1}(#2)}
\newcommand{\inversibleMatrixSpace}[2]{GL_{#1}(#2)}
\newcommand{\spanspace}[1]{\left<#1\right>}
% Group theory

\def\isomorphe{\simeq}
\newcommand{\ordergroup}[1]{|#1|}
\newcommand{\GSautomorphismDef}[2]{Aut_{\text{#1}}(#2)}
\newcommand{\generatedGroup}[1]{\left<#1\right>}

% Field theory

%extensionDegree
\newcommand{\extensionDegree}[2]{[#1 : #2]}

\newcommand{\extensionField}[2]{#1/#2}

%plongement
\newcommand{\plongement}[3]{Hom_{#1}(#2, #3)}

%Galois group
\newcommand{\galoisGroup}[2]{G(#1, #2)}

% Spectrum Theory
\newcommand{\spectrum}[1]{\sigma(#1)}
\newcommand{\resolvant}[1]{\rho(#1)}

\def\Ldeux{\mathcal{L}^{2}}
\def\Ldeuxstar{(\mathcal{L}^{2})^{*}}

%GSsequence :
%		#1 : represention of elements of the sequences
%		#2 : indices
%		#3 : set definition
\newcommand{\GSsequence}[3]{(#1_{#2})_{#2 \in #3}}

%GSsetDef :
%		#1 : set elements
\newcommand{\GSset}[1]{\left\{ #1 \right\}}

%GSsetDef :
%		#1 : global set
%		#2 : condition
\newcommand{\GSsetDef}[2]{\left\{#1 \, | \, #2 \right\}}

%GSprodSet :
%		#1 : indice
%		#2 : begin indice
%		#3 : end indice
%		#4 : set
\newcommand{\GSprodSet}[4]{\displaystyle \prod_{#1 = #2}^{#3} #4_{#1}}

%GSsum :
%		#1 : indice
%		#2 : begin indice
%		#3 : end indice
%		#4 : element
\newcommand{\GSsum}[4]{\displaystyle \sum_{#1 = #2}^{#3} #4}

\newcommand{\GSintervalCC}[2]{\left[#1, #2\right]}
%-------------------------------------------------------------------------------

%Analysis :

% conjuguate
\def\conjuguate{\overline}

% miLength: multi-indice length
\newcommand{\miLength}[1]{|#1|}

% segment: all points between two points given
\newcommand{\segment}[2]{S(#1, #2)}

%GSfunction :
%       #1 : name function
%       #2 : begin set
%       #3 : end set
\newcommand{\GSfunction}[3]{#1 : #2 \rightarrow #3}

%GSnorme : Deprecated --> \norm
%		#1 : elements which norme is applied on
\newcommand{\GSnorme}[1]{\norm{#1}}

%GSnormeDef :
%		#1 : elements which norme is applied on
%		#2 : norme indice
\newcommand{\GSnormeDef}[2]{\norm{#1}_{#2}}

%GSnormedSpace :
%		#1 : vectorial space
%		#2 : \GSnorme[Def] with dot as element.
\newcommand{\GSnormedSpace}[2]{(#1, #2)}

%Identification
\def\identification{\simeq}

%GSdual
%		#1 : vectorial space
\newcommand{\GSdual}[1]{#1^{*}}

%GSbidual
%		#1 : vectorial space
\newcommand{\GSbidual}[1]{#1^{**}}

\newcommand{\GSunitBoule}[1]{\mathcal{B}_{#1}}
\newcommand{\GSclosedUnitBoule}[1]{\adh{\GSunitBoule{#1}}}

\newcommand{\GSweakTopo}[1]{\sigma(#1, #1^{*})}
\newcommand{\GSpreweakTopo}[1]{\sigma(#1^{*}, #1)}

%GSendomorphism
\newcommand{\GSendomorphism}[1]{End(#1)}

%GShomomorphisme
\newcommand{\GShomomorphisme}[3][]
{
	Hom_{#1}(#2, #3)
}

%GShomomorphismeDef
% Deprecated !! Use instead directly \GShomomorsphisme
\newcommand{\GShomomorphismeDef}[3][]
{
	\GShomomorphisme{#1}{#2}{#3}
}

%GScontinueEndo
\newcommand{\GScontinueEndo}[2][]
{
	\mathcal{L}_{#1}(#2)
}

%\GScontinueHomo
\newcommand{\GScontinueHomo}[3][]
{
	\mathcal{L}_{#1}(#2; #3)
}

%\GScompactEndo
\newcommand{\GScompactEndo}[1]{\mathcal{K}(#1)}

%\GScompactHomo
\newcommand{\GScompactHomo}[2]{\mathcal{K}(#1; #2)}

\newcommand{\GSfiniteRankHomo}[2]{\mathcal{R}_{f}(#1; #2)}
\newcommand{\GSfiniteRankEndo}[1]{\mathcal{R}_{f}(#1)}

\newcommand{\GSisomorphisme}[1]{Isom(#1)}
\newcommand{\GSisomorphismeHomo}[2]{Isom(#1; #2)}
\newcommand{\GSisometryEndo}[1]{Isom(#1)}

\newcommand{\jacobienneMatrix}[2]{J_{#1}(#2)}
\newcommand{\hessienneMatrix}[2]{\mathcal{H}_{#1}(#2)}
%-------------------------------------------------------------------------------

%Model theory

\def\La{\mathcal{L}}
\def\Th{\mathcal{T}}
\def\SA{\mathcal{A}}
\def\SB{\mathcal{B}}

%Ultraproduct
%	1 : indice elements
%	2 : set which contains indices
%	3 : ultrafilter
%	4 : models represention
\newcommand{\GSultraproduct}[4]{\displaystyle {\prod_{#1 \in #2}}^{#3}#4_{#1}}

%Ultrapower
%	1 : indice elements
%	2 : set which contains indices
%	3 : ultrafilter
%	4 : model
\newcommand{\GSultrapower}[4]{\displaystyle {\prod_{#1 \in #2}}^{#3} #4}

%Substructures
\newcommand{\GSsubstructure}[2]{#1 \subseteq #2}

%Elementary Substructures.
\newcommand{\GSelemSubstructure}[2]{#1 \preceq #2}

%Elementary equivalent structures
\newcommand{\GSelemEquivStructure}[3]{#2 \equiv_{#1} #3}
%-------------------------------------------------------------------------------

%Hilbert space
\def\Hilbert{\mathcal{H}}
\newcommand{\GSortho}[1]{#1^{\perp}}
\def\GSid{\cong}
\newcommand{\dotprod}[2]{\bra{#1}\ket{#2}}
\newcommand{\adjointe}[1]{#1^{*}}
%-------------------------------------------------------------------------------

%Group representions
\newcommand{\GSrepr}[2]{Repr(#1, #2)}
\newcommand{\GSreprf}[2]{Repr_{f}(#1, #2)}
\newcommand{\GSrepri}[2]{Repr_{i}(#1, #2)}
%-------------------------------------------------------------------------------

\usepackage{amsfonts}
\usepackage{amssymb}
\usepackage{amsmath}
\usepackage{amsthm}
\usepackage{mathrsfs}

\newtheorem{definition}{Définition}

\newtheorem{proposition}[definition]{Proposition}
\newtheorem{lemma}[definition]{Lemme}
\newtheorem{corollary}[definition]{Corollaire}
\newtheorem{theorem}[definition]{Théorème}

\newtheorem{exemple}{Exemple}
\newtheorem*{question}{Questions}
\newtheorem*{remarque}{Remarque}

\newtheorem{exercice}{Exercice}
\newtheorem*{notation}{Notation}


\title{Théorie des modèles I}
\author{}

\begin{document}

\maketitle

\section{Définition}

\begin{definition} [Complexité d'une formule]
	Soient $\lang{L}$ un langage, et $\phi$ une $\lang{L}$-formule.

	On définit \textbf{la complexité de $\phi$} comme le nombre de
	quantificateurs universels ou existentiels contenus dans la formule.  Une
	formule sans quantificateurs est de complexité $0$.
\end{definition}

\begin{definition} [$\kappa$-catégorique]
	Soient $\lang{L}$ un langage et $\theory{T}$ une $\lang{L}$-théorie consistante.

	Soit $\kappa$ un cardinal infini.

	On dit que $\theory{T}$ est \textbf{$\kappa$-catégorique} si tous les
	modèles de $\theory{T}$ de cardinalité $\kappa$ sont isomorphes,
	c'est-à-dire que tous les modèles de cardinalité $\kappa$ répondent aux
	mêmes théories, ie satisfont les mêmes formules.
\end{definition}

\begin{proposition}
	La théorie des $\rational$ espaces vectoriels est $\aleph_{1}$-catégorique.
\end{proposition}

\begin{proof}

\end{proof}

\begin{definition} [Groupe divisible]
	Soit $G$ un groupe abélien. On dit que $G$ est \textbf{divisible} si pour
	tout $n \in \naturel_{0}$, pour tout $g \in G$, il existe $h \in G$ tel que
	$g = \underbrace{h \cdots h}_{n \text{ fois}}$.
\end{definition}

La théorie des groupes abéliens divisibles comportent une liste dénombrable
d'axiomes (au premier ordre). En effet, il faut citer, pour chaque $n$, l'axiome
cité dans la définition.

\begin{proposition}
	Soient $\theory{T}_{D}$ la théorie des groupes abéliens divisibles, et
	$\theory{T}_{\rational}$ la théorie des $\rational$-espaces vectoriels.

	Alors, si $\struct{A}$ est un modèle de $\theory{T}_{D}$, alors $\struct{A}$ est
	un modèle de $\theory{T}_{\rational}$.

	En d'autres termes, tout groupe divisible abélien peut être vu comme un
	$\rational$-espace vectoriel.
\end{proposition}

\begin{proof}

\end{proof}

\begin{definition} [Modèle-complète]
	Soit $\lang{L}$ un langage, et $\theory{T}$ une $\lang{L}$-théorie.
	On dit que \textbf{$\theory{T}$ est modèle-complète} si toute sous-structure
	est élémentaire, c'est-à-dire, si pour tous $\struct{A}$, $\struct{B}$ tel
	que $\struct{A} \substructure \struct{B}$, on a $\struct{A}
	\elemSubstructure \struct{B}$.
\end{definition}

\begin{definition} [Existentiellement clos]
	Soit $\lang{L}$ un langage.

	Soient $\struct{A}$, $\struct{B}$ deux $\lang{L}$-structures tel que
	$\struct{A} \substructure \struct{B}$.

	On dit que \textbf{$\struct{A}$ est existentiellement clos dans
	$\struct{B}$}, noté $\struct{A} \existentiallyClosed \struct{B}$, si toute
	formule existentielle à paramètres dans $\struct{A}$ est vraie dans
	$\struct{A}$ et dans $\struct{B}$.
\end{definition}

\section{Théorèmes de Lowenheim Skolem}

\subsection{Enoncés et preuves}

\begin{theorem} [Test de Tarski-Vaught]
	\label{theorem:tarski_vaught}
	Soit $\lang{L}$ un langage.

	Soient $\struct{A}$, $\struct{B}$ deux $\lang{L}$-structures tel que
	$\struct{A} \substructure \struct{B}$.

	Alors, les assertions suivantes sont équivalentes.

	\begin{itemize}
		\item $\struct{A} \elemSubstructure \struct{B}$
		\item Pour toute $\lang{L}$-formule $\phi(x, \vec{y})$ où $\vec{y} =
			(y_{1}, \dots, y_{m}) $, pour tout uple $\vec{a} = (a_{1}, \dots,
			a_{m}) \in A^{m}$, si $\struct{B} \models \exists x \, \phi(x,
			\vec{a})$, alors $\exists b \in A, \struct{A} \models \phi(b,
			\vec{a})$.
	\end{itemize}
\end{theorem}

\begin{proof}

\end{proof}

Donnons maintenant le théorème de Lowenheim Skolem (descendant).

\begin{theorem} [Théorème de Lowenheim-Skolem descendant]
	Soit $\lang{L}$ un langage dénombrable.

	Soit $\struct{A}$ une $\lang{L}$-structure, et $E \subseteq A$.

	Alors il existe une sous-structure élémentaire $\struct{A}_{0}$ de
	$\struct{A}$ contenue dans $A$ tel que $\cardinal{A_{0}} \leq
	\aleph_{0}$.
	\label{theorem:lowenheim_skolem_descendant}
\end{theorem}

Ce théorème est très important, et assez impressionnant. En effet, il signifie
que si on prend une quelconque structure sur un langage donné, nous pouvons
trouver une structure au plus dénombrable qui répond aux mêmes formules que
notre structure de départ. C'est-à-dire que nous pouvons trouver des structures
aussi petites qu'on veut répondant à la théorie de notre structure initiale.

\begin{proof}

\end{proof}

Nous obtenons également une version montante, dans le sens que nous pouvons
aussi trouver des structures aussi grandes qu'on souhaite qui garde la même théorie.

\begin{theorem} [Théorème de Lowenheim-Skolem montant]
	Soit $\lang{L}$ un langage dénombrable.

	Soit $\struct{A}$ une $\lang{L}$-structure, et $\kappa$ tel que
	$\GSset{\cardinal{A}, \aleph_{0}} \leq \kappa$ un cardinal.

	Alors il existe une structure $\struct{B}$ tel que $\struct{A}
	\elemSubstructure \struct{B}$ tel que $\cardinal{B} = \kappa$.
	\label{theorem:lowenheim_skolem_montant}
\end{theorem}

\begin{proof}

\end{proof}


\subsection{Applications}

Nous obtenons directement un corollaire des théorèmes de Lowenheim-Skolem.

\begin{corollary}
	\label{lemma:complete_equiv_elementaire_equivalentes}
	Soit $\lang{L}$ un langage, et soit $\theory{T}$ une $\lang{L}$-théorie.

	Alors les assertions suivantes sont équivalentes.

	\begin{itemize}
		\item $\theory{T}$ est complète.
		\item Tous les modèles de $\theory{T}$ sont élémentairement équivalents.
	\end{itemize}
\end{corollary}

\begin{proof}

\end{proof}

\begin{exemple}
	La théorie des ordres denses n'est pas complète car on peut construire la
	théorie des ordres denses avec extrémités, et la théorie des ordres denses
	sans extrémités, dont les modèles ne sont pas tous élémentairement
	équivalents. Par exemple, si on prend le modèle de la théorie des ordres
	denses avec comme ensemble de base $[0, 1]$, nous n'avons pas qu'il est
	élémentairement équivalent au modèle sur $]0, 1[$ car il existe un minimum dans
	le premier, et non dans le second.
\end{exemple}

\begin{theorem} [Théorème de Vaught]
	\label{theorem:vaught}
	Soient $\lang{L}$ un langage, et $\theory{T}$ une $\lang{L}$-théorie
	consistante, $\kappa$-catégorique ne
	comportant que des modèles infinis.

	Alors $\theory{T}$ est complète.
\end{theorem}

\begin{proof}

\end{proof}

Remarquons que la condition que $\theory{T}$ ne possède que des modèles infinis
est essentiels. En effet, s'il possède un modèle fini, de cardinalité $n$ par exemple, alors la formule:

\begin{equation}
	\phi(x) := \exists a_{1}, \cdots, \exists a_{n} x = a_{b}
\end{equation}

est vraie dans le modèle fini, mais non dans un modèle infini. Un exemple est
la théorie des groupes abéliens dont tous les éléments sont d'ordre $2$, qui est
laissé en exercice.


\section{Elimination des quantificateurs}

\begin{definition}
	\label{definition:elimination_quantificateurs}
	Soit $\lang{L}$ un langage, et soit $\theory{T}$ une $\lang{T}$-théorie.

	On dit que $\theory{T}$ a \textbf{l'élimination des quantificateurs} si
	pour toute $\lang{L}$-formule $\phi(\vec{x})$, il existe une
	$\lang{L}$-formule sans quantificateurs $\theta(\vec{x})$ tel que

	\begin{equation}
		T \satisfies [\forall \vec{x}, (\phi(\vec{x}) \equiv \theta(\vec{x}))]
	\end{equation}
\end{definition}

\begin{exemple}
	\begin{itemize}
		\item La théorie des corps algébriquement clos ($ACF$) possède l'élimination des
	quantificateurs.
		\item Sur le langage $\lang{L} = (+, -, ., <, 0, 1)$, la théorie $T :=
			Th_{\lang{L}}(\real, +, -, ., <, 0, 1)$ (interprétation usuelle) a
			l'élimination des quantificateurs.
		\item Sur le langage $\lang{L} = (+, -, ., 0, 1)$, la théorie $T :=
			Th_{\lang{L}}(\real, +, -, ., 0, 1)$ (interprétation usuelle) n'a pas
			l'élimination des quantificateurs.
	\end{itemize}
\end{exemple}

Par le dernier exemple, nous remarquons que le langage a de l'importance.

La définition que nous avons donnée pour l'élimination des quantificateurs est
une définition purement syntaxique: il n'y a aucune structure qui apparaît.
On pourrait se demander s'il est possible d'obtenir un critère équivalent
purement sémantique.

Nous sommes alors mené au théorème suivant.

\begin{theorem}
	\label{theorem:critere_eq_formule_par_formule}
	Soit $\lang{L}$ un langage comportant au moins un symbole de constante, et
	soit $\theory{T}$ une $\lang{L}$-théorie consistante.

	Soit $\phi(\vec{x})$ une $\lang{L}$-formule.

	Alors les assertions suivantes sont équivalentes.

	\begin{itemize}
		\item Il existe une formule sans quantificateurs $\theta(\vec{x})$ tel
			que
			\begin{equation}
				T \satisfies [\forall \vec{x}, (\phi(\vec{x}) \equiv
				\theta(\vec{x}))]
			\end{equation}
		\item Pour tous modèles $\struct{A}$, $\struct{B}$ de $\theory{T}$,
			toutes sous-structures $\struct{C}$ de $\struct{A}$ et $\struct{B}$,
			et tout $\vec{a} \subseteq C$, on a
			\begin{equation}
				\struct{A} \models \phi(\vec{a}) \equiv \struct{B} \models
				\phi(\vec{a})
			\end{equation}
	\end{itemize}
\end{theorem}

\begin{proof}

\end{proof}

\begin{proposition}
	Soit $\lang{L}$ un langage, et soit $\theory{T}$ une $\lang{L}$-théorie
	modèle-complète. Alors $\theory{T}$ a l'élimination des quantificateurs.
\end{proposition}

\begin{proof}

\end{proof}

\begin{remarque}
	La réciproque n'est pas vraie. En effet, la théorie $Th_{\lang{L}}(\real, +,
	-, ., 0 1)$ est modèle-complète mais n'a pas l'élimination des
	quantificateurs.
\end{remarque}

\section{Ensemble définissable}


\section{Exercices}

\begin{exercice}
	La théorie des ordres denses sans extrémités est $\aleph_{0}$-catégorique.
\end{exercice}

\begin{exercice}
	La théorie des groupes abéliens dont tous les éléments sont d'ordre $2$
	n'est pas complète mais est $\aleph_{0}$ catégorique.
\end{exercice}

\begin{exercice}
	La théorie $Th_{\lang{L}}{(\real, +, -, ., 0, 1)}$ où $\lang{L} = (+, -, .,
	0, 1)$ n'a pas l'élimination des quantificateurs.
\end{exercice}

\begin{exercice}
	La théorie $Th_{\lang{L}}{(\real, +, -, ., 0, 1)}$ où $\lang{L} = (+, -, .,
	0, 1)$ est modèle-complète.
\end{exercice}
\end{document}


